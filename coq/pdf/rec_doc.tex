\documentclass{article}

\usepackage[normalem]{ulem}
\usepackage{amsmath,amssymb}
\usepackage[margin=1in]{geometry}
\usepackage{multicol}
\usepackage{ottalt}
\usepackage{xcolor}
\usepackage{xspace}
\usepackage{amssymb}

\inputott[ott]{record}

\newcommand\calculus{$\mathsf{F}_i^+$\xspace}
\newcommand\reccal{$\lambda^{\text{rec}}$\xspace}

\title{Elaborating \calculus to record calculus \reccal}
\author{Snow}

\begin{document}

\maketitle


\paragraph{Changes in \reccal}
0) Simplify the system, drop polymorphism, and use simple labels\\
1) Type soundness proved\\
2) Use list for record types, so the time cost of concatenation can be
moved to compilation.

\section{Syntax of \reccal}

\ottgrammartabular{
  \ottttyp
}

\ottgrammartabular{
  \otttexp
}


\paragraph{Notes}
0) Multi-field records are nullable.
1) The whole program evaluates to a record form.
3) $l$ stands for a string.
%% 2) As any expression can be concatenated, all values should have a record form.
%% For example, we cannot evaluates ${ ->Base => \x.b }.->Base$ in $ [-];{Base=>b} $.
%% It is hard to express lazy evaluation then.
%% Our evaluation strategy is eager but when constructing records, each field is
%% in a thunk so they will not be evaluated until projection.

%% \paragraph{Questions}
%% (/\ X * Top . \ x : X . x : X)


\ottdefnsValues

\ottdefnsTargetStep

\ottdefnsConcatTypes

\ottdefnsRecordTypes

\ottdefnsWelformedTypes

\ottdefnsConcatTypes

\ottdefnsTargetTyping


%%%%%%%%%%%%%%%%%%%%%%%%%%%%%%%%%%%%%%%%%%%%%
\newpage
\section{Syntax of \calculus}

\ottgrammartabular{
  \otttyp
}

\ottgrammartabular{
  \ottexp
}


\paragraph{Notes}
0) Fixpoint typing rule updated. \\
1) Type translation (from source type to string) modeled.\\
2) Current focus: converting source type to target type.

% \ottdefnsEqIndexType

\subsection{Elaboration rules}
1) All top-like values are translated into empty lists.

\begin{mathpar}
  \ottdruleElaXXTop{} \and
  \ottdruleElaXXTopAbs{} \and
  \ottdruleElaXXTopRcd{}
\end{mathpar}

2) All basic values (that are not merges) are translated into single-field records.

\begin{mathpar}
  \ottdruleElaXXBase{} \and
  \ottdruleElaXXAbs{} \and
  \ottdruleElaXXRcd{}
\end{mathpar}

3) Variables are directly used (unwrap the thunk) and merged terms are simply concatenated,
with the expectation that $ \cdot   \vdash  \ottnt{e} \, \Rightarrow \, \ottnt{A}  \rightsquigarrow  \ottnt{t}$ implies $\ottnt{t}  \rightarrow^{*}  \ottsym{\{}   | \ottnt{A} |   \Rightarrow  \ottnt{t'}  \ottsym{\}}$
(or a multi-field record) (i.e. the variable will be substituted
by a term of this exact type).

\begin{mathpar}
  \ottdruleElaXXVar{} \and
  \ottdruleElaXXMerge{}
\end{mathpar}

4) Projection, application, and type application make use of the property
that every translated subterms are records. Application is lazy.

\begin{mathpar}
  \ottdruleElaXXProj{} \and
  \ottdruleElaXXApp{}
\end{mathpar}

5) Annotated expressions, like fixpoints, rely on the subsumption rule to
insert coercions. The coerced expression always evaluates to a record.

\begin{mathpar}
  \ottdruleElaXXAnno{} \and
  \ottdruleElaXXFix{} \and
  \ottdruleElaXXSub{}
\end{mathpar}

\subsection{Coercions}

\ottdefnsCoSubtyping

\ottdefnsCoMerge

\ottdefnsDistributiveApplication

\subsection{Auxiliary definitions}

%% \boxed{ | \ottnt{A} |  =  \top } \hfill \textit{(Type translation)}
\begin{mathpar}
%% \inferrule{}{  |  \mathbb{B}  |  =  \mathbb{B}  }
%% \and
%% \inferrule{}{ |A->B| = ->|B| }
%% \and
%% \inferrule{}{ |{l:A}| = {l:|A|} }
%% \and
\inferrule
{ A_{k_1} < A_{k_2} < \cdots < A_{k_m} \\ \neg \rceil A_k \lceil }
{ |A_1\&A_2\&\cdots\&A_n| = |A_{k_1}|\&|A_{k_2}|\&\cdots\&|A_{k_m}| }
\end{mathpar}

\bigskip\bigskip

\ottdefntoplike

\ottdefnsOrdinaryType

\ottdefnsDisjoint

\ottdefnsSplitType

\ottdefnsTargetSubtype

% \ottdefnsTargetEqIndexType

\begin{verbatim}
1) type soundness of the target calculus

2) connecting the source to the target

- type translation function (from source type to string):
intersection types are flattened, their
toplike parts are filtered, and all the conjuncts are sorted (but no deduplication).

- a relation that characterizes types that corresponds to the same
string.

3) I need a function that converts source types to target types, to prove that
elaboration generates well-typed terms. There are 3 problems:
- Source types and target types are not one-to-one mapped. The typing rule for
toplike lambda functions does not split types but the subsumption rule does
(via subtyping)
\x:Int . x : Top&Top ~> { }
(\x:Int . x : Int) : Int->Top&Top ~> { ->|Top&Top| => \ x . ... }


\end{verbatim}

Lemma eqIndTyp\_sound\_complete: \quad $A ~= B$ iff $|A| = |B|$.\\

Lemma disjoint\_type\_no\_eqInd: \quad if $A ~= B$ and $A * B$ then
there is a contradiction.\\

Lemma translation from source type to target type is a function \\

Lemma cosub\_well\_typed: \quad if $\ottnt{t_{{\mathrm{1}}}}  :  \ottnt{A}  \ottsym{<:}  \ottnt{B}  \rightsquigarrow  \ottnt{t_{{\mathrm{2}}}}$ and $ G || \vdash t1 <= |[A]|$ then $ G || \vdash t2 : |[B]|$.\\

Theorem elaboration\_well\_typed: \quad if $\Gamma  \vdash  \ottnt{e} \, \ottnt{dirflag} \, \ottnt{A}  \rightsquigarrow  \ottnt{t}$ then
$|[ G ]|| \vdash t : |[A]|$.\\

\end{document}
